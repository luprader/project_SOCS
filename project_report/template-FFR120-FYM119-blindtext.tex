% ****** Start of file template-FFR120-FYM120-blindtext.tex ******
%
% use on Overleaf!!!!
%
\documentclass[%
 reprint,
 amsmath,amssymb,
 aps,
]{revtex4-2}

\usepackage{graphicx}% Include figure files
\usepackage{dcolumn}% Align table columns on decimal point
\usepackage{bm}% bold math
\usepackage{hyperref}% add hypertext capabilities
\usepackage{array} % for better column types
\usepackage{caption} % optional for better captions
\usepackage{tabularx}
%\usepackage[mathlines]{lineno}% Enable numbering of text and display math
%\linenumbers\relax % Commence numbering lines
\usepackage{xcolor}

\usepackage{lipsum}

\begin{document}

\title{[T2.1] Characterizing Shortest-Path Ensembles for Brain Network Modeling \\ Using Empirical Constraints}% Force line breaks with \\

\author{Daniel Peña Fonseca}
\author{Lukas Prader}

\date{\today}% It is always \today, today,
             %  but any date may be explicitly specified

\begin{abstract} %%% DO NOT CHANGE!
%%% - B1 - %%%%%%%%%%%%%%%%%%%%%%%%%%%%%%%%%%% 
%%% Customize this part: text between - B1 - and - E1 - must not appear in the final report 
\noindent
Understanding how anatomical pathways manage the flow of information in the human brain remains a fundamental challenge in (network) neuroscience. Parametric models, particularly the Shortest Path Ensembles (SPE), provide a framework for simulating communication between different regions of interest (ROIs) not only through the shortest route, but also through a set of alternatives (near-shortest paths) based on model parameters. Nevertheless, the number and relative weighing of these near-shortest paths that would best approximate empirical results, remain unknown. Conceptually, a small number of allowed paths captures the biological intuition that brain communication makes use of structural information to use optimal routing, while a large number of allowed paths describes a transmission without the use of any encoded structural information. Using structural connectomes and then simulating signal outputs over time, this report addresses this question by varying the number of paths considered to determine which range minimizes discrepancies between empirical data (connectomes and functional MRIs) and simulated results.
%%% - E1 - %%%%%%%%%%%%%%%%%%%%%%%%%%%%%%%%%%%%%%

\begin{description} %%% DO NOT CHANGE!
\item[Project Topic] %%% DO NOT CHANGE!
{Brain Network Modelling} %CHANGE accordingly
\item[Teaching Assistant] %%% DO NOT CHANGE!
{Yu-Wei Chang} % CHANGE accordingly
\end{description} %%% DO NOT CHANGE!
\end{abstract}

\maketitle




\section{\label{sec:intro}Introduction} %%% DO NOT CHANGE!

%%% - B2 - %%%%%%%%%%%%%%%%%%%%%%%%%%%%%%%%%%% 
%%% Customize this part: text between - B2 - and - E2 - must not appear in the final report 
\noindent
\fbox{\parbox[t][][t]{\columnwidth}{
\textbf{Points:} \fbox{\bf \textcolor{magenta}{5}} out of \textbf{44}

In this section you have to: 
Provide the reader of the general context of the research question that you chose to investigate, providing the relevant peer-reviewed references.  For example, a citation looks like \cite{surname1999peer} or \cite{cognome2001what, smullyan2025what}.

Remember: No subsectioning in the introduction.

For references, use the command $\backslash${\tt cite}$\{\cdots\}$. \\
Example: $\backslash${\tt cite}$\{${\tt polya1945how}$\}$ gives \cite{polya1945how}. 
Use the bib file {\tt biblio-FFR120-FYM119.bib} (provided with this template) for the bib items.
}}

\textcolor{cyan}{\lipsum[4-6]}
%%% - E2 - %%%%%%%%%%%%%%%%%%%%%%%%%%%%%%%%%%%%%%






\section{\label{sec:overview}Overview} %%% DO NOT CHANGE!

%%% - B3 - %%%%%%%%%%%%%%%%%%%%%%%%%%%%%%%%%%% 
%%% Customize this part: text between - B3 - and - E3 - must not appear in the final report 
\noindent
With the goal of exploring the relation of structural connections between brain regions to functional correlations in brain activity, connectome data can be used to simulate brain activity using a range of Brain Network Communication Models (BNCMs) \cite{review_methods}. While routing protocol based models assume that the network operates with a lot of information about its structure and mainly communicates through the most efficient (shortest) paths; diffusion models assume that information propagates without any use of structural information, taking more inefficient paths with higher metabolic transmission cost. So-called parametric models are located between these extremes, attempting to use rules that can balance the amount of information that is used in transmissions. Examples of this type of frameworks include linear threshold models, biased random walks and shortest path ensembles. A common property among them is their ability to be tuned via a parameter, measuring how far the model should lean towards diffusion-type models or routing protocol approaches.

% They all have in common that they can be tuned using a parameter

\begin{table*}[t]
\caption{\textbf{Overview of simulation methods and metrics.}}
\label{tab:methodsoverview}

% I dont like what I have written in the table, it is hard to write stuff for these kinds of "scenarions" etc

\begin{tabular*}{\textwidth}{|c|c|c|c|}
    \hline
    \textbf{Method / Model} & \textbf{Use case scenario} & \textbf{Features} & \textbf{Suitable for the project?} \\
    \hline
    \parbox[t]{0.125\linewidth}{Linear threshold model} &
    \parbox[t]{0.305\linewidth}{looking at signal propagation based on the state of neighbouring nodes} &
    \parbox[t]{0.19\linewidth}{neighbour transmission threshold} &
    \parbox[t]{0.318\linewidth}{uses activation state to determine propagation, focused on direct structural influence} \\
    \hline
    \parbox[t]{0.125\linewidth}{Biased random walk} & 
    \parbox[t]{0.305\linewidth}{investigating the potential impact of structural information on random walk dynamics} &
    \parbox[t]{0.19\linewidth}{transition bias, amount of structural information allowed to influence random transitions} &
    \parbox[t]{0.318\linewidth}{can be used to study the concrete relationship between structural properties and patterns in time correlation} \\
    \hline
    \parbox[t]{0.125\linewidth}{Shortest path ensembles} & 
    \parbox[t]{0.305\linewidth}{analysing behaviour related to perturbations of the optimal path and the relevance of the optimal path in relation to other alternatives} &
    \parbox[t]{0.19\linewidth}{number of next shortest paths $k$} &
    \parbox[t]{0.318\linewidth}{can be used to closely explore differences in efficiency between paths related to, but deviating from optimal (ideal) path routing} \\
    \hline
\end{tabular*}

\end{table*}


{\bf Method 1 (Linear threshold models).} {In this approach, signals are transmitted based on the number of neighbours $n$ that have received a signal already. The number necessary for a signal to be transmitted is the parameter that can be varied in this model. If $n=0$ the system resembles free broadcasting diffusion through the system, every node propagates the signal to its neighbours. With higher values the range of reachable nodes decreases until it becomes very localised for large values of $n$.}

{\bf Method 2 (Biased random walks).} {In this framework, the transmissions are modelled as random walks with transition probabilities between nodes being biased depending on the topological properties, i.e. their distance to the shortest path considering a specific target node. Depending on the amount of information used to bias the transition, the result will resemble that of a purely random walk or a shortest path trajectory.}

{\bf Method 3 (Shortest path ensembles).} {Instead of only using the shortest path to some target node, one can consider all k-shortest paths, which are the ones most closely resembling the optimal path. The number $k$ of considered paths determines if the system more closely resembles that of shortest path routing or random walk paths.}

%%% - E3 - %%%%%%%%%%%%%%%%%%%%%%%%%%%%%%%%%%%%%%




\section{\label{sec:method}Method} %%% DO NOT CHANGE!

%%% - B4 - %%%%%%%%%%%%%%%%%%%%%%%%%%%%%%%%%%% 
%%% Customize this part: text between - B4 - and - E4 - must not appear in the final report 
\begin{figure*}
    \centering
    \includegraphics[width=\textwidth]{Fig1_placeholder.pdf}
    \caption{{\bf Method employed (change this text according to your project).} Write some explanatory text for the figure. Make sure that the figure is referenced in the text. The caption should be such that, together with the figure, allows the reader to understand the concepts in the figure itself. 
    \textcolor{blue}{\bf A figure describing the method/model is MANDATORY.} Missing to include a figure causes a {\bf deduction} (-3  points). 
    } 
    \label{fig:selectedmethod}
\end{figure*}

\noindent
In order to study a wide range of behaviour with one model, the use of a parametric model was chosen as the brain network communication model. In particular, it was decided to use the shortest path ensemble approach. Several metrics will be used to characterize the behaviour of this communication method.
{\bf Shortest path ensembles (SPE).} {This model proposes that a signal propagating from a one region to another, can only do so  through one of the \textit{k} shortest paths (ensemble) \cite{kshortest_original}. Following the SPE model, each route $i$ will be assigned a probability $P_i$ of the signal communicating through it according to its length $d_i$ \cite{kprobabilities}. To assign higher probabilities to shorter routes, probabilities will be defined as follows:
\begin{equation}
    P_i=\frac{1/d_i}{\sum_{i=1}^k (1/d_i)}
\end{equation}

This parametric model allows an analysis dependent on the size of the ensemble. Modifying the number of shortest paths considered, \textit{k}, the metrics presented afterwards will be studied, in order to find the range of \textit{k}-values for which the empirical results are most faithfully reproduced. 

While other processes generally present high energy or information costs, SPE not only preserves low delay costs, but also prevents other costs from rising.
In addition, SPE allows to compare the presence and impact of time delay by comparing paths within the ensemble to each other, making it possible to explore the structural properties which shape the propagation of signals in the proximity of efficient (shortest path) and inefficient (detour) communication routes.

[Discuss the use case scenarios of the method, their features with advantages and disadvantages / strong and weak points, and whether it is more or less suitable for the research question. It is a good idea to reference the literature.]} 

{\bf Functional Correlation (FC).} {In brain-network modelling, this procedure is commonly used to measures the correlation structure of activity time series across brain regions, generally using an FC matrix \cite{general_methods}. Where each element of the matrix is the Pearson correlation coefficient between the time series of two brain regions. With this metric, an exhaustive comparison between empirical outcomes and simulated results will be performed. Although this measure does not cover all the characteristics of a brain's network, it works as a general reference for different ROIs' correlation.


% We could say the good and bad parts after describing the methods.



[Discuss the use case scenarios of the method, their features with advantages and disadvantages / strong and weak points, and whether it is more or less suitable for the research question. It is a good idea to reference the literature]

{\bf Recurrent Quantification Analysis (RQA).} {
With fMRIs in the form of time series, one can perform analyses on the recurrence of temporal patterns between ROIs. Recurrence plots and information theoretic measures can be used to compare different time series with each other \cite{webber2015recurrence} and will be used to compare simulated results with empirical data.
These tools usually require domain specific knowledge in order to find good parameter values for the analysis.
}

{\bf Shannon Entropy Growth Curve.} {
The Shannon entropy growth curve for a time series of states of a dynamical systems can be used to understand the temporal order of complexity of a system, showing the exact orders of time relevant to the behaviour of the system \cite{crutchfield2003regularities}. It calculates the block entropies for a time series and looks at changes in the growth behaviour.
It has been used in applications related to Reinforcement Learning and general state estimation in Markov Processes.
The possible estimation length of the growth curve depends highly on the system and the number of samples, since a naive maximum likelihood estimation is biased to underestimate the entropy.}
%%% - E4 - %%%%%%%%%%%%%%%%%%%%%%%%%%%%%%%%%%%%%%





\section{\label{sec:results}Results and Discussion} %%% DO NOT CHANGE!

%%% - B5 - %%%%%%%%%%%%%%%%%%%%%%%%%%%%%%%%%%% 
%%% Customize this part: text between - B5 - and - E5 - must not appear in the final report 
\noindent
\fbox{\parbox[t][][t]{\columnwidth}{
\textbf{Points:} \fbox{\bf \textcolor{magenta}{8}} out of \textbf{44}

Here report and discuss the results of the project. 

Remember to organize your results properly.  

Reference your figures in the discussion. Example: Fig.~\ref{fig:res1}. 
}}

\textcolor{cyan}{\lipsum[1-2]}\\

\begin{figure}
    \centering
    \includegraphics[width=\columnwidth]{Fig3.pdf}
    \caption{{\bf Title Figure.} Write some explanatory text for the figure. Make sure that the figure is referenced in the text. The caption should be such that, together with the figure, allows the reader to understand the concepts in the figure itself.} 
    \label{fig:res1}
\end{figure}

\textcolor{cyan}{\lipsum[8-9]} 

At some point in the discussion, Reference Fig.~\ref{fig:res2}. \\

\begin{figure}[h]
    \centering
    \includegraphics[width=\columnwidth]{Fig2.pdf}
    \caption{{\bf Title Figure.} Write some explanatory text for the figure. Make sure that the figure is referenced in the text. The caption should be such that, together with the figure, allows the reader to understand the concepts in the figure itself.} 
    \label{fig:res2}
\end{figure}

\textcolor{cyan}{\lipsum[10-11]} 
%%% - E5 - %%%%%%%%%%%%%%%%%%%%%%%%%%%%%%%%%%%%%%




\section{\label{sec:conclusion}Conclusions and Outlook} %%% DO NOT CHANGE!

%%% - B6 - %%%%%%%%%%%%%%%%%%%%%%%%%%%%%%%%%%% 
%%% Customize this part: text between - B6 - and - E6 - must not appear in the final report 
\noindent
\fbox{\parbox[t][][t]{\columnwidth}{
\textbf{Points:} \fbox{\bf \textcolor{magenta}{4}} out of \textbf{44}

Here conclusions and outlook. 
}}

\textcolor{cyan}{\lipsum[1]}

\textcolor{cyan}{Lorem ipsum dolor sit amet, consectetuer adipiscing elit. Aenean commodo ligula eget dolor. Aenean massa. Cum sociis natoque penatibus et magnis dis parturient montes, nascetur ridiculus mus. Donec quam felis, ultricies nec, pellentesque eu, pretium quis, sem.  }
%%% - E6 - %%%%%%%%%%%%%%%%%%%%%%%%%%%%%%%%%%%%%%





\section{\label{sec:Contribution}Contributions} %%% DO NOT CHANGE!

%%% - B7 - %%%%%%%%%%%%%%%%%%%%%%%%%%%%%%%%%%% 
%%% Customize this part: text between - B7 - and - E7 - must not appear in the final report 
\noindent
\fbox{\parbox[t][][t]{\columnwidth}{
\textbf{Points:} \fbox{\bf \textcolor{magenta}{1}} out of \textbf{44}

Mandatory in all cases (also in the case of 1-person team). 

Add some text. See a possible example text below.
}}
Example: A.B. and C.D. conceived and implemented ... (adapt). E.F., G.H., contributed to .... etc etc. All authors ... (adapt).
%%% - E7 - %%%%%%%%%%%%%%%%%%%%%%%%%%%%%%%%%%%%%%



\section{\label{sec:COI}Conflict of Interest} %%% DO NOT CHANGE!

%%% - B8 - %%%%%%%%%%%%%%%%%%%%%%%%%%%%%%%%%%% 
%%% Customize this part: text between - B8 - and - E8 - must not appear in the final report 
\noindent
\fbox{\parbox[b][][t]{\columnwidth}{
\textbf{Points:} \fbox{\bf \textcolor{magenta}{1}} out of \textbf{44}

Add some text. See the the following Note for a clarification.
}}

{{\bf Note:} A conflict of interest can also be known as {\emph competing interest}. A conflict of interest can occur when you, or your employer, or sponsor have a financial, commercial, legal, or professional relationship with other organizations, or with the people working with them, that could influence your research.
For example check here:} \href{https://authorservices.taylorandfrancis.com/editorial-policies/competing-interest/}{https://authorservices.taylorandfrancis.com/editorial-policies/competing-interest/}
%%% - E8 - %%%%%%%%%%%%%%%%%%%%%%%%%%%%%%%%%%%%%%




\section{\label{sec:datacode}Data and Code Availability} %%% DO NOT CHANGE!

%%% - B9 - %%%%%%%%%%%%%%%%%%%%%%%%%%%%%%%%%%% 
%%% Customize this part: text between - B9 - and - E9 - must not appear in the final report 
\noindent
\fbox{\parbox[b][][t]{\columnwidth}{
\textbf{Points:} \fbox{\bf \textcolor{magenta}{1}} out of \textbf{44}

Add some text. See a possible text below.
}}

Data is available for download at ....  / can be accessed from ....

All source code and examples are made publicly available at ... . The version used in this study is archived in ... with DOI ... 
%%% - E9 - %%%%%%%%%%%%%%%%%%%%%%%%%%%%%%%%%%%%%%



%%% - B10 - %%%%%%%%%%%%%%%%%%%%%%%%%%%%%%%%%%% 
%%% Customize this part: text between - B10 - and - E10 - must not appear in the final report 
\noindent
\fbox{\parbox[b][][t]{\columnwidth}{
Score for correct amount of relevant, peer reviewed {\bf References}: 
\textbf{Points:} \fbox{\bf \textcolor{magenta}{5}} out of \textbf{44}
}}


\bibliography{biblio-FFR120-FYM119} %%% DO NOT CHANGE!
% Produces the bibliography via BibTeX.

\end{document}
