% ****** Start of file template-FFR120-FYM120-blindtext.tex ******
%
% use on Overleaf!!!!
%
\documentclass[%
 reprint,
 amsmath,amssymb,
 aps,
]{revtex4-2}

\usepackage{graphicx}% Include figure files
% \usepackage{dcolumn}% Align table columns on decimal point
\usepackage{bm}% bold math
\usepackage{hyperref}% add hypertext capabilities
% \usepackage{array} % for better column types
\usepackage{caption} % optional for better captions
\usepackage{tabularx}
%\usepackage[mathlines]{lineno}% Enable numbering of text and display math
%\linenumbers\relax % Commence numbering lines
\usepackage{xcolor}
\usepackage[american]{babel}

\usepackage{lipsum}

\begin{document}

\title{[T2.1] Characterizing Shortest-Path Ensembles for Brain Network Modeling \\ Using Empirical Constraints}% Force line breaks with \\

\author{Daniel Peña Fonseca}
\author{Lukas Prader}

\date{\today}% It is always \today, today,
             %  but any date may be explicitly specified

\begin{abstract} %%% DO NOT CHANGE!
%%% - B1 - %%%%%%%%%%%%%%%%%%%%%%%%%%%%%%%%%%% 
%%% Customize this part: text between - B1 - and - E1 - must not appear in the final report 
\noindent
Understanding how anatomical pathways manage the flow of information in the human brain remains a fundamental challenge in (network) neuroscience. Parametric models, particularly the Shortest Path Ensembles (SPE), provide a framework for simulating communication between different Regions of interest (ROIs) not only through the shortest route, but also through a set of alternatives (near-shortest paths) based on model parameters. The number and relative weighing of these near-shortest paths that would best approximate empirical results remain unknown. Conceptually, a small number of allowed paths captures the biological intuition that brain communication makes use of structural information to use optimal routing, while a large number of allowed paths describes a transmission without the use of any encoded structural information. Using structural connectomes and then simulating signal outputs over time, this report addresses this question by varying the number of paths considered to determine which range minimizes discrepancies between empirical data (connectomes and functional MRIs) and simulated results.
%%% - E1 - %%%%%%%%%%%%%%%%%%%%%%%%%%%%%%%%%%%%%%

\begin{description} %%% DO NOT CHANGE!
\item[Project Topic] %%% DO NOT CHANGE!
{Brain Network Modelling} %CHANGE accordingly
\item[Teaching Assistant] %%% DO NOT CHANGE!
{Yu-Wei Chang} % CHANGE accordingly
\end{description} %%% DO NOT CHANGE!
\end{abstract}

\maketitle




\section{\label{sec:intro}Introduction} %%% DO NOT CHANGE!

%%% - B2 - %%%%%%%%%%%%%%%%%%%%%%%%%%%%%%%%%%% 
%%% Customize this part: text between - B2 - and - E2 - must not appear in the final report 

Understanding how information propagates between brain regions is a central challenge in neuroscience.
Structural connectivity, as mapped through Diffusion Tensor Imaging (DTI), encodes the pathways that information can flow through when propagating from one Region of Interest (ROI) to another.
Multiple studies have shown that the strength and structure of white-matter routes are tightly linked to functional coupling along the cortex \cite{white_matter}.
Network neuroscience offers a quantitative framework for studying these connections, using graph approaches to characterize how the brain's topology supports efficient, flexible and metabolically economical communication \cite{book_graph_brain}.

Decades of research have shown that the human connectome is neither random nor uniformly organized.
Instead, it exhibits characteristic properties such as small-world topology, combining high local clustering with additional long-range connections \cite{small_world_brain_networks}.
The network also contains highly connected hub regions, creating a structure supporting long-range communication, essential for efficient information propagation \cite{rich_club_original}\cite{rich_club_2}.
These architectural principles indicate that the connectome is optimized for dynamic information transfer, while operating under biological constraints, such as wiring cost and metabolic efficiency.

Against this structural background, an important challenge is to determine how information flows through the network; that is, from the brain's anatomical structure, determine how brain nodes are functionally correlated, which can be measured through functional MRI (fMRI).
Communication dynamics have been proposed as the missing link between structural and functional connectivity, providing mechanisms of how neural systems coordinate across spatially separated regions \cite{kshortest_original}.
Classical frameworks often assume either optimal routing, where signals follow the most efficient (shortest) path, or diffusive dynamics, where information flows stochastically according to local connectivity \cite{review_methods}.
% Particularly, random walk processes have been deeply explored, and serve as intuitive models for diffusion-like communication in neural systems \cite{random_walk}.

However, both extremes capture only certain aspects of real neural communication, and place assumptions about the nature of information flow.
In shortest-path routing, signals are assumed to have full knowledge of the global network structure, while diffusive models assume that regions have only local information to guide communication \cite{review_methods}.
Brain network topology actually achievs a balance between diffusion and routing efficiency due to the highly connected hubs present in the connectome \cite{review_brain_structure_and_function}.

Parametric models operate on the continuum between the diffusive and shortest-path routing extremes \cite{review_methods}, striking a trade-off between transmission cost and information cost.
Additionally, studies have shown that including alternative pathways can improve efficiency and strengthen robustness and resilience in brain networks \cite{review_brain_structure_and_function}, which are additional factors to consider for realistic models of neural communication.

Despite these insights, major gaps remain.
In particular, there is no consensus over which kinds of paths or how many should be used for a signal to travel between ROIs.
Seeking to answer these questions, in this study we investigate information transfer using a parametric model to capture the parameter values for which results from simulations based on structural connectomes most accurately reflect empirical patterns of brain activity.



\section{\label{sec:overview}Overview} %%% DO NOT CHANGE!

%%% - B3 - %%%%%%%%%%%%%%%%%%%%%%%%%%%%%%%%%%% 
%%% Customize this part: text between - B3 - and - E3 - must not appear in the final report 
\noindent


\begin{table*}[!ht]
\caption{\label{tab:methodsoverview} \textbf{Overview of simulation methods and metrics.}}

\begin{tabular*}{\textwidth}{|c|c|c|c|}
    \hline
    \textbf{Method / Model} & \textbf{Use case scenario} & \textbf{Features} & \textbf{Suitable for the project?} \\
    \hline
    \parbox[t]{0.125\linewidth}{Linear threshold model} &
    \parbox[t]{0.305\linewidth}{looking at signal propagation over time while influenced by local neighbourhood activation} &
    \parbox[t]{0.19\linewidth}{neighbour transmission threshold} &
    \parbox[t]{0.318\linewidth}{can characterize communication models based on transmission time and neighbourhood topology} \\
    \hline
    \parbox[t]{0.125\linewidth}{Biased random walk} & 
    \parbox[t]{0.305\linewidth}{investigating the impact of structural information on random walk dynamics} &
    \parbox[t]{0.19\linewidth}{transition bias, amount of structural information allowed to influence random transitions} &
    \parbox[t]{0.318\linewidth}{classifies models by the amount of structural information necessary to select paths compared to random walks} \\
    \hline
    \parbox[t]{0.125\linewidth}{Shortest path ensembles} & 
    \parbox[t]{0.305\linewidth}{analysing behaviour related to deviations from optimal path signalling} &
    \parbox[t]{0.19\linewidth}{number of next shortest paths $k$} &
    \parbox[t]{0.318\linewidth}{differentiates models simply by the number of almost optimal paths allowed for communication} \\
    \hline
\end{tabular*}

\end{table*}

Connectome data can be used to simulate brain activity based on a range of Brain Network Communication Models (BNCMs) \cite{review_methods}.\
When wanting to explore the impact of available information on the performance of communication models, parametric models provide a flexible framework due to their tunable parameters.
They enable specific parameter tuning, which can give insights into how strongly different trade offs (e.g. efficiency vs transmission cost) should be considered to match empirical observations.
Examples of this type of framework include linear threshold models \cite{linear_threshold}, biased random walks \cite{biased_random_walk} and shortest path ensembles \cite{kshortest_original}.
A common property among them is their ability to tune how far the model should lean towards diffusion-type models or shortest-path approaches by changing parameters.

In order to determine the method most suitable for our research question, the features and specific use case scenarios for each of them need to be compared (see Table \ref{tab:methodsoverview}).

{\bf Method 1 (Linear threshold models).} 
In this approach, signal propagation is determined by the number of neighbouring nodes that have already received the signal \cite{linear_threshold}.
Each node in the network will receive the information only if a proportion of more than $\theta$ of its neighbours has previously received the transmission, making $\theta$ the tunable parameter of the model.
With $\theta=0$, each node with at least one activated neighbour will receive the signal, resulting in maximal diffusion through broadcasting \cite{review_methods}.
Increasing $\theta$, the conditions for activation become progressively stricter, reducing the propagation throughout the network.
For $\theta$ sufficiently large, information flow can reach a bottleneck, reflecting a regime in which only tightly clustered neighbourhoods can participate in transmission.


{\bf Method 2 (Biased random walks).} 
In unbiased random walks, the probability of a signal propagating from node $i$ to node $j$ is proportional to the connection weight between these nodes \cite{random_walk}.
Alternatively, in biased random walks, transition probabilities are additionally influenced by the amount of global topological information available to individual nodes \cite{biased_random_walk}.
This is controlled by the tunable parameter $\lambda$. With $\lambda=0$, unbiased random walks are recovered, since individual nodes will only posses information regarding their own connections.
In contrast, for $\lambda \to \infty$, the biased random walk converges to the shortest path, since each transmission between every two nodes will contain the entire global topological information, leading to optimal paths \cite{book_graph_brain}.


{\bf Method 3 (Shortest path ensembles).} 
This parametric model considers that signals between two brain regions do not only travel along the most efficient route, but through the $k$ most efficient paths (sorted by length) \cite{kshortest_original}.
Standard shortest-path routing is recovered setting $k=1$, while $k \to \infty$ implies diffusive communication since all possible paths would be considered.
Small values of $k$, would require extensive global topological knowledge of the network, while larger values of $k$ would relax these conditions, resulting in smaller informational cost \cite{review_methods}.
Since SPE is a relaxation of shortest path routing, many metrics and methods applicable to shortest path routing can be adapted to SPE.

All three methods can be considered suitable for the research question, as they allow to change the kinds of paths allowed for communication by varying their respective parameters.

Linear threshold models focus on the influence of local neighbourhoods and the amount of time (steps) it takes for information to spread from one ROI to another.
This makes them suitable to investigate the communication speed and efficiency of transmissions spreading over time.

Several metrics, like diffusion efficiency and search information, use random walks as their basis, making biased random walks suitable for studies focusing on these metrics.
They focus on the efficiency and accessibility of communication paths, linking global structural properties to functional connectivities.

SPEs are inherently more selective, as they focus on the most efficient paths and their near alternatives \cite{review_methods}.
The related $k$-shortest path length measures the distance between ROIs based on the lengths of the $k$-shortest paths, weighted by their accessibility compared to all other options on the path \cite{kshortest_original}.
This creates a distance metric which also captures the accessibility or ease of traversal of the chosen paths, with a tight link to optimal path topology.

Since we are interested in how tuning a parametric model changes the types of paths used in order to best match empirical results, the SPE model was chosen.
Since $k$ determines the number of considered paths, it is easy to interpret compared to other, more abstract metrics (search information, diffusion efficiency) which may be necessary to explain results using biased random walks.
% We were also not interested in time-resolved propagation dynamics, making linear threshold models less suitable.


%%% - E3 - %%%%%%%%%%%%%%%%%%%%%%%%%%%%%%%%%%%%%%




\section{\label{sec:method}Method} %%% DO NOT CHANGE!

%%% - B4 - %%%%%%%%%%%%%%%%%%%%%%%%%%%%%%%%%%% 
%%% Customize this part: text between - B4 - and - E4 - must not appear in the final report 
\begin{figure*}[!ht]
    \centering
    \includegraphics[width=\textwidth]{workflow.png}
    \caption{{\bf Method overview.} The illustrated workflow summarizes the steps taken to simulate and evaluate the shortest-path ensemble model against empirical brain data for all patients. Structural connectivity matrices and fMRI time series from the provided dataset are used as inputs.
    SC matrices are used to derive the k-distance matrices for $k$ ranging from 1 to 50. The obtained distance matrices are then used as coupling matrices in the Kuramoto model to simulate brain activity over time, resulting in simulated FC matrices. Empirical FC matrices are computed from the fMRI time series using Pearson correlation. Finally, the simulated FC matrices are compared to the empirical FC matrices using Pearson correlation to assess model performance across different values of $k$.
    }
    \label{workflow}
\end{figure*}


\noindent
We developed a workflow (Fig. \ref{workflow}) to evaluate the performance of the SPE model while varying $k$ by simulating brain activity and comparing it to empirical data using the Python programming language.

{\bf Dataset.}
The provided dataset contains structural connectivity (SC) matrices obtained with measurements from Diffusion Tensor Imaging (DTI) time series from resting-state functional Magnetic Resonance Imaging (rfMRI) data from 100 healthy participants \cite{braindata}.
The measurements concern 246 brain ROIs defined by the Brainnetome atlas \cite{fan2016brainnetome}.
% (see Fig. \ref{fig:brain_atlas_views} for a visualization of the atlas centroids).
% \begin{figure}
%     \centering
%     \includegraphics[width=0.9\columnwidth]{../project_SOCS_code/figures/brain_atlas_views.png}
%     \caption{\textbf{Brain atlas views.} Visualization of the Brainnetome atlas showing the 246 brain regions of interest (ROIs).}
%     \label{fig:brain_atlas_views}
% \end{figure}

The SC matrices are sparse, symmetric matrices of the mean fractional anisotropy (values between 0 and 1) between ROIs representing the connection strength between them (Fig. \ref{fig:avg_sc_fc} A).
The time series represent the averaged  Blood Oxygen Level Dependent (BOLD) signal from each ROI, z-scored over time for each ROI and thus having mean 0 and standard deviation 1.
Time steps of 0.72 seconds were used for 4800 measurements, which is equivalent to 57.6 minutes.

\begin{figure}[!ht]
    \centering
    \includegraphics[width=\columnwidth]{../project_SOCS_code/figures/averaged_sc_fc.png}
    \caption{\textbf{Averaged SC and FC matrix.} Matrices showing the averaged structural connectivity (A) and functional connectivity (B) over all patients. Color describes the connection strength or functional connectivity between two ROIs respectively.}
    \label{fig:avg_sc_fc}
\end{figure}

{\bf K-shortest path length.}
Making use of the SPE model, we compute the k-shortest path length \cite{kshortest_original} between every pair of ROIs for each patient.

For this, we compute the edge lengths between ROIs as $l_{u,v} = -\log(w_{u,v})$, where $w_{u,v}$ is the connection weight between nodes $u$ and $v$ in the SC matrix.
Using logarithmic transform creates edge lengths that increase the number of paths that are able to be incorporated in efficient communication.

The topological length $d(\pi_{u,v})$ of a path $\pi_{u,v}$ between nodes $u$ and $v$ is then defined as the sum of the edge lengths along the path.
The \texttt{scipy} \cite{scipy} implementation of Yen's algorithm was used to compute the $k$ shortest paths between every pair of ROIs, for $k$ up to 50.

The k-shortest path length $D_k(s,t)$ between nodes $s$ and $t$ is then defined as the weighted average of the lengths of the $k$ shortest paths between them:

\begin{equation}
    D_k(s,t) = \sum_k \hat{P}(\pi_{s,t}^k)  d(\pi_{s,t}^k).
    \label{eq:kdist}
\end{equation}

$\hat{P}(\pi_{s,t}^k)$ is the normalized probability $P(\pi_{s,t}^k)$ of the k-shortest path $\pi_{s,t}^k$ being followed under random walk dynamics:

\begin{equation}
    P(\pi_{s,t}^k) = \sum_{e_{u,v} \in \pi_{s,t}^k} \frac{w_{u,v}}{\sum_j w_{u,j}}.
    \label{eq:path_prob}
\end{equation}

$D_k(s,t)$ thus represents the combined distances of all k-shortest paths, weighted by their accessibility under random walk dynamics.
For each patient and each value of $k$ from 1 to 50, we get a k-distance matrix containing the k-shortest path lengths between every pair of ROIs.

{\bf Functional Connectivity (FC).} 
Derived from ROI activity, functional connectivity describes the statistical relationships between ROI time series \cite{review_brain_structure_and_function}.
The simplest way is to calculate the Pearson correlation coefficient between the time series of each pair of ROIs, resulting in a symmetric FC matrix (Fig. \ref{fig:avg_sc_fc} B).
For each patient, we calculated the empirical FC matrix from the provided fMRI time series and from the simulated ROI activiy using Pearson correlation.

{\bf Kuramoto Model.}
In order to simulate brain activity based on the empirical brain structure we decided to model the brain regions as a network of coupled Kuramoto oscillators.
It was chosen since it is simpler, yet with similar performance to more biophysically realistic (i.e. neural mass) models \cite{kuramoto-eq}, while still being more flexible than simple firing rate models, which use a decaying firing rate under influence of random noise and input from the network \cite{general_methods}.

Defining $\theta(t)$ as the phase (frequency) of node $n \in \{ 1,\dots, N \}$ at time $t$, the dynamics of the system are goverend by the following equation \cite{kuramoto-eq}:

\begin{equation}
    \frac{d \theta_n}{dt} = \omega_n + c\cdot \sum_{p=1}^{N} C_{np}\cdot \sin{ \Big( \theta_p (t-\tau_{np}) -\theta_n(t) \Big) } + \eta_n(t).
\end{equation}

The term $\omega_n$ represents the intrinsic angular frequency of each ROI, satisfying $\omega_n=2\pi f_n$, where $\nu_n$ is initially sampled from a fixed Gaussian distribution with mean $f_0 = 60 Hz$ and standard deviation $\sigma_f = 5$.
$C_{np}$ is the relative coupling strength from node $p$ to node $n$, and $c$ is the global coupling strength, which scales all connection strengths.
The delay in transmissions from $p$ to $n$ is incorporated through $\tau_{np}$, which depends on the distance between the nodes.
Defining $L_{np}$ as one element of the distance matrix, and the average distance between nodes as $\langle L\rangle$, the delay is defined as $\tau_{np}=\langle \tau \rangle L_{np} /\langle L\rangle$, with $\langle \tau \rangle =11$ms.
Lastly, $\eta_n(t)$ represents noise present in the network, implemented as Gaussian white noise with mean 0 and standard deviation $\sigma_n = 3$.
All parameter values were taken from \cite{kuramoto-eq}.

In our case, the distance matrix $L$ is one of the k-distance matrices, while the coupling matrix $C$, also called adjaciency matrix, is calculated using $C_{np}=1/L_{np}$ $\forall n,p$.

This framework allows us to simulate brain activity time series for each patient and each value of $k$, evolving the phases of all ROIs over time and finally obtaining activity time series by calculating $\sin(\theta_n(t))$.


{\bf Simulation.}
Due to the high computational cost of simulating the systems, we only simulated for a selection of $k$ values, namely $k \in \{ 1, 2, \dots, 10, 25, 35, 50 \}$.
Simulations were performed for 5 seconds using a time step of 0.1 ms.
FC matrices were calculated from the simulated activity after removing the first 500ms to allow the system to stabilize.

{\bf Evaluation.}
In order to evaluate the SPE model performance, we calculated the Pearson correlation between the upper triangle elements, excluding the diagonal.
These correlations were then analyzed to determine the best performing $k$ values on a mean and per-patient basis.
To check for significant differences between different $k$ values, Analysis of Variance (ANOVA) tests were performed, which check if the distributions between samples (here all simulations for a particular $k$) are significantly different.

{\bf Parameter Optimization.}
The global coupling strength $c$ needs to be of the correct scale to ensure stability and validity of the system \cite{kuramoto-eq}.
With more connections in the coupling matrix, the value of $c$ has to be smaller \cite{general_methods}.
\cite{kuramoto-eq} proposes a value of $c = 18$ when using SC matrices with 66 ROIs as adjaciency matrices.
Since our coupling matrices have 246 ROIs and are not directly comparable to SC matrices, we performed a parameter sweep to find the optimal scaling factor $s$ for $c = s \cdot 18$.
For a particular value of $s$, simulations of 2 seconds using a stepsize of 0.1ms were performed for $k=1,5,10,30$ with three repetitions each. Upper triangle pearson correlations were calculated between simulated and empirical FC matrices and then averaged over all $k$ and repetitions to get a performance value.

Initial tests suggested the performance function to have a unimodal maximum, which is why it was decided to optimize $s$ by minimizing the negative performance function with \texttt{scipy's} \cite{scipy} implementation of Brent's method in the bounded interval $[0.05, 1]$, iterating for 10 steps.
It uses a combination of bisection, secant method and parabolic interpolation to find the minimum of a function without requiring derivatives, making it suitable for our optimization problem where the performance function may not be smooth or differentiable.


%%% - E4 - %%%%%%%%%%%%%%%%%%%%%%%%%%%%%%%%%%%%%%





\section{\label{sec:results}Results and Discussion} %%% DO NOT CHANGE!

%%% - B5 - %%%%%%%%%%%%%%%%%%%%%%%%%%%%%%%%%%% 
%%% Customize this part: text between - B5 - and - E5 - must not appear in the final report 

Figure \ref{fig:kdist_panel} shows three example k-distance matrices ($k=1,5,25$), obtained by averaging over all patients. It can be seen that for increasing values of $k$, the matrices become more uniform in color. This suggests that once many paths are considered, ROIs become more similar in terms of their k-distance, as signals can propagate through many alternative routes, creating a diffusive regime. For lower values of $k$, the matrices show more structure, indicating that certain ROIs form well connected clusters, i.e. the regions around the index 150 and 200 (Fig. \ref{fig:kdist_panel}, top).

\begin{figure}
    \centering
    \includegraphics[width=0.9\columnwidth]{../project_SOCS_code/figures/averaged_k_dist_matrices.png}
    \caption{\textbf{Example k-distance matrices.} Three k-distance matrices showing k=1, 5 and 25, obtained by averaging over all patients. Color describes the k-distance between two ROIs.}
    \label{fig:kdist_panel}
\end{figure}


In order to show how the structure of the k-distance matrices relates to empirical functional connectivity, the Pearson correlation between both matrices was calculated for each patient.
The results were then plotted as the mean correlation and standard deviation as a function of $k$ (Fig. \ref{fig:kdist_vs_fc}).
The first thing to notice is the negative sign of the correlation values.
Since we correlate distance with functional connectivity, this implies that ROIs with higher k-distance between them exhibit lower functional connectivity.
One can also observe clear k-dependence, with the strongest mean correlation at $k=1$ with $r \approx -0.129$, which is equivalent to single shortest path connections between ROIs.
For $k>1$ correlation is significantly lower, but has a stable plateau with $r \approx -0.125$ up to $k=10$, after which correlation decreases further, reaching $r \approx -0.1$ at $k=50$.
This implies that additional paths, if close to the shortest path, relate differently to functional connectivity than more distant paths.
The scale of the standard deviation indicates that there is considerable variability between patients, yet ANOVA confirms significant differences between the distributions of each $k$ with $p \ll 0.001$.

\begin{figure}[!ht]
    \centering
    \includegraphics[width=\columnwidth]{../project_SOCS_code/figures/k_dist_correlation_vs_k.png}
    \caption{\textbf{Correlation of k-distance matrices and empirical FC.} Pearson correlation between k-distance matrices and empirical functional connectivity matrices for all patients, shown as the mean with standard deviation as error band.}
    \label{fig:kdist_vs_fc}
\end{figure}



The optimization of the scaling factor $s$ scaling the global coupling strength $c$ is shown in Figure \ref{fig:parameter_opt}, with the maximal correlation between simulated and empirical FC achieved with a scaling factor $f \approx 0.2098$.
% The $k$ values used for optimization (1,5,10,30) were chosen to put emphasis on the region of interest $k \leq 10$ and keep computation time low.

\begin{figure}[!ht]
    \centering
    \includegraphics[width=\columnwidth]{../project_SOCS_code/figures/parameter_optimisation.png}
    \caption{\textbf{Parameter optimisation.} Optimisation results of the scaling parameter for the Kuramoto global coupling factor. Colored points show results for different values of k, each point is the mean of 3 repetitions. Black crosses show the average correlation of the tested k values.}
    \label{fig:parameter_opt}
\end{figure}

The optimal scaling factor scales the coupling strength to about one fifth of the original value, which is close to the ratio of the number of ROIs used in \cite{kuramoto-eq} (66) and the Brainnetome atlas (246).
This confirms that the optimal coupling strength depends on the number of connections in the connectivity matrix.


With the optimized scaling factor, simulations for a selection of $k$ values were performed for all patients and resulted in a set of simulated FC matrices for each patient.
Example simulations again using the averaged k-distance matrices over all patients for $k=1,5,25$ (Fig. \ref{fig:sim_fc_panel}) show clear differences compared to the empirical averaged FC (Fig. \ref{fig:avg_sc_fc} B).
While regions of high correlation like around index 200 are captured by the simulations, the size and distribution of these regions differ significantly.
It can also be observed that with increasing $k$, the simulated FC matrices become more uniform in color, similar to the k-distance matrices (Fig. \ref{fig:kdist_panel}).
Regions of high correlation in simulated FC generally seem to have matching regions of low distance in the k-distance matrices.
Direct pearson correlations between the averaged k-distance matrices and the averaged simulated FC matrices returned correlations of $-0.355$ for $k=1$, $-0.486$ for $k=5$, $-0.347$ for $k=25$, showing that even though there are similarities in structure, the simulated FC matrices differ significantly from the k-distance matrices and provide different information.

The simulated FC matrices were also compared to the empirical FC matrices per patient using Pearson correlation (Fig. \ref{fig:sim_vs_fc}).

\begin{figure}
    \centering
    \includegraphics[width=0.9\columnwidth]{../project_SOCS_code/figures/averaged_simulated_fc_matrices.png}
    \caption{\textbf{Example k-distance matrices.} Three simulated FC matrices showing k=1, 5 and 25, obtained by averaging over all patients. Color describes the k-distance between two ROIs.}
    \label{fig:sim_fc_panel}
\end{figure}

Results from simulation contrast those obtained from direct correlation of k-distance matrices with empirical FC (Fig. \ref{fig:kdist_vs_fc}), suggesting that actually not shortest path communication, but communication using k-distance coupling with $k=8$ achieves the highest (now positive since we correlate simulated FC with empirical FC) mean correlation with empirical FC at $r \approx 0.056$. The shortest path coupling ($k=1$) even has the lowest average correlation of all simulated $k$, only reaching $r = 0.032$
This indicates that actual brain dynamics may not rely solely on shortest path communication, but rather include a higher single digit combination of near-shortest paths.

\begin{figure}
    \centering
    \includegraphics[width=\columnwidth]{../project_SOCS_code/figures/simulation_correlation_vs_k.png}
    \caption{\textbf{Correlation of simulated FC and empirical FC.} Pearson correlation between simulated functional connectivity matrices and empirical functional connectivity matrices for all patients, shown as the mean with standard deviation as error bars.}
    \label{fig:sim_vs_fc}
\end{figure}


In order to further investigate the variability between patients, the best $k$ value (highest correlation) was determined for each patient, both for the direct k-distance to empirical FC correlation and for the simulated to empirical FC correlation.
These were then plotted as histograms (Fig. \ref{fig:best_k_hist}).
For the direct correlation of k-distance matrices with empirical FC (Fig. \ref{fig:best_k_hist} A), the distribution of best $k$ values peaks strongly at $k=1$, though with some individuals having optimal correlation around $k=8$.
One individual even has an optimal correlation at$k=49$, showing how different results can be even in healthy individuals.
For the simulated to empirical FC correlation (Fig. \ref{fig:best_k_hist} B), the distribution is more spread out than in the previous case.

\begin{figure}
    \centering
    \includegraphics[width=0.9\columnwidth]{../project_SOCS_code/figures/best_k_histograms.png}
    \caption{\textbf{Distribution of best k per patient.} Histograms showing the distribution of best k values per patient, for the correlation between k-distance matrices and empirical FC (A) and for the correlation between simulated FC and empirical FC (B).}
    \label{fig:best_k_hist}
\end{figure}

In individual counts, $k=10$ actually has the highest number of patients with optimal correlation, though followed closely by $k=8$, which had the highest average correlation (Fig. \ref{fig:sim_vs_fc}). 15 individuals have their optimal correlation at $k=25, 35$ or $50$, again showing the high variability between individuals.

%%% - E5 - %%%%%%%%%%%%%%%%%%%%%%%%%%%%%%%%%%%%%%

\section{\label{sec:conclusion}Conclusions and Outlook} %%% DO NOT CHANGE!

%%% - B6 - %%%%%%%%%%%%%%%%%%%%%%%%%%%%%%%%%%% 
%%% Customize this part: text between - B6 - and - E6 - must not appear in the final report 
\noindent
\fbox{\parbox[t][][t]{\columnwidth}{
\textbf{Points:} \fbox{\bf \textcolor{magenta}{4}} out of \textbf{44}
}}


{\color{red} k distance always is the weighted value of all paths up to k, but maybe one should also just look at other combinations of paths, maybe k3 is not used but 4 and 6 etc. this is a limitation of what we did

kuramoto is a simplification, and we will not have simulated for the same timescales as the empirical measurements, this may be a problem with comparability, though I did ensure that the oscillators had stabilized within the time frame that i simulated

also in actuality we should convert the data that we get to BOLD signal, I have some code from a guy that supposedly does this, maybe i can still add that.}

The optimal parameters curently obtained for the model vary strongly depending on the k-distance matrix used, but they are mostly in the same order of magnitude as the ones proposed in \cite{kuramoto-eq}.

Further investigation will be conducted to see if the model can be improved to obtain statistically significant results.

The base correlation of the k-distance matrices against empirical FC is interesting due to the suggested additional importance of paths with length $\approx$ 5 in addition to shortest path signalling, which has the highest correlation.
%%% - E6 - %%%%%%%%%%%%%%%%%%%%%%%%%%%%%%%%%%%%%%





\section{\label{sec:Contribution}Contributions} %%% DO NOT CHANGE!

%%% - B7 - %%%%%%%%%%%%%%%%%%%%%%%%%%%%%%%%%%% 
%%% Customize this part: text between - B7 - and - E7 - must not appear in the final report 
\noindent
Lukas Prader read the articles regarding the information propagation and implemented all the code needed for this project. Daniel P. Fonseca red  the articles regarding previous research in the field and wrote this article. Both authors, nevertheless, made valuable contributions for the complementary work. \textit{Note:} Due to a mistake when creating the GitHub repository, all of Daniel's contributions have been assigned to Lukas' profile.
%%% - E7 - %%%%%%%%%%%%%%%%%%%%%%%%%%%%%%%%%%%%%%



\section{\label{sec:COI}Conflict of Interest} %%% DO NOT CHANGE!

%%% - B8 - %%%%%%%%%%%%%%%%%%%%%%%%%%%%%%%%%%% 
%%% Customize this part: text between - B8 - and - E8 - must not appear in the final report 
\noindent
The authors declare that the research was conducted in the absence of any commercial or financial relationships that could be construed as a potential conflict of interest.
%%% - E8 - %%%%%%%%%%%%%%%%%%%%%%%%%%%%%%%%%%%%%%




\section{\label{sec:datacode}Data and Code Availability} %%% DO NOT CHANGE!

%%% - B9 - %%%%%%%%%%%%%%%%%%%%%%%%%%%%%%%%%%% 
%%% Customize this part: text between - B9 - and - E9 - must not appear in the final report 
\noindent
Data is available for download at \cite{braindata}. All source files (code and report) are made publicly available at \cite{code} .
%%% - E9 - %%%%%%%%%%%%%%%%%%%%%%%%%%%%%%%%%%%%%%



%%% - B10 - %%%%%%%%%%%%%%%%%%%%%%%%%%%%%%%%%%% 
%%% Customize this part: text between - B10 - and - E10 - must not appear in the final report 
\noindent



\bibliography{biblio-FFR120-FYM119} %%% DO NOT CHANGE!
% Produces the bibliography via BibTeX.

\end{document}
